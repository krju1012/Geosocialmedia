\section{Problem Statement}

As consequence of the climate change, natural hazards are occurring with a larger quantity. Other than the amount of occurring disasters such  as floods, the increasing population density in inner cities support the vulnerability of inhabitants. While the threat of natural hazards are becoming more relevant, the amount of collected data, especially user-generated content, is increasing significantly but not commonly used for disaster management systems. \\
With the beginning of Web 2.0 and its user-driven information social aspects became more and more popular within the internet. Social-Media platforms provide a framework of sharing content. For the proposed research the image hosting platform Flickr is used. While Flickr users are spread commonly over developed and emerging countries the additional data could be used to improve the accuracy of existing flood-related disaster management systems as an supplementary approach of data gathering. \\
This approach of \textit{citizens as sensors} collects vast amounts of spatial data in realtime whereas further filtering methods needs to be applied. For areas with lack of information of sensor infrastructure, web users can provide useful data, volunteered or in ambient form, which adds significant content to otherwise unknown areas in terms of available data. In this case, with research for flood events, a possible scenario could be missing/ failing water level sensors at a waterflow but several Flickr posts of users with flood-related tags in this certain area of interest. Therefore, this additional information can lead disaster managers to further knowledge about existing floods, for example where flooded areas occurred and at which time certain regions were influenced. \\
In summary, following research questions need to be addressed:
\begin{enumerate}
\item To which extent are flood intensities analysable with help of Flickr messages?
\item Can training methods of historic floods improve the accuracy of flood coverage analysis?
\item Is the theoretical approach of Social-Media based flood analysis feasible to realise in practice?
\item Which parameters does a region need to be suited for flood analysis with AGI?
\end{enumerate}

	